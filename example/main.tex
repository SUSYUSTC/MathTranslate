
\documentclass[UTF8]{ctexart}
\usepackage{amsmath,amssymb}
\begin{document}

这封信检验了戴森-施温格 (DS) 方程作为量子场论计算工具的有效性。 DS 方程是耦合方程的无限序列,这些方程完全满足场论的连通格林函数 \( G_{n} \)。这些方程将较低的格林函数与较高的格林函数联系起来,如果它们被截断,则所得的有限方程系统是欠定的。求解欠定系统的最简单方法是将所有更高的格林函数设置为零,然后求解前几个格林函数的结果确定系统。可以将如此获得的 \( G_{1} \) 或 \( G_{2} \) 与可求解模型中的精确结果进行比较,以查看高阶截断的准确性是否有所提高。研究了五个 \( D=0 \) 模型:Hermitian \( \phi^{4} \) 和 \( \phi^{6} \) 以及非 Hermitian \( i \phi^{3},-\phi^{4} \) 和 \( i \phi^{5} \) 理论。截断的 DS 方程给出了一系列近似值,这些近似值缓慢收敛到一个极限值,但这个极限值总是与精确值相差百分之几。基于类似平均场的近似值的更复杂的截断方案不能解决这个令人生畏的计算问题。
量子场论的目标是计算连通格林函数\( G_{n}\left(x_{1}, \ldots, x_{n}\right) \),它包含了理论的物理内容。程序原则上是求解\( \phi(x) \)场的场方程,然后计算\( \phi: \gamma_{n}\left(x_{1}, \ldots, x_{n}\right) \equiv\left\langle 0\left|\mathrm{~T}\left\{\phi\left(x_{1}\right) \ldots \phi\left(x_{n}\right)\right\}\right| 0\right\rangle \)时序产品的真空期望值。然后将未连接的格林函数 \( \gamma_{n} \) 组合成累积量以获得 \( G_{n} \) [1]。

Dyson-Schwinger (DS) 方程旨在通过使用 \( c \) 数泛函分析来计算 \( G_{n} \) 的扰动和非扰动行为,而不求助于算子理论 \( [2-5] \)。该过程是将耦合 DS 方程的无限系统截断为一组有限的耦合方程,这些方程可以很好地近似前几个 \( G_{n} \)。问题在于,虽然 DS 方程完全由 \( G_{n} \) 满足,但 DS 方程是一个欠定系统;每个新方程都会引入额外的格林函数,因此系统的截断包含比方程更多的格林函数 [6]。一个合理的策略是通过将最高格林函数设置为零来关闭截断系统,然后求解得到的确定系统。在这里,我们研究最简单的情况:零维时空中的量子场论。逐次消除给出 \( G_{1} \) 或 \( G_{2} \) 的多项式方程。随着五个 \( D=0 \) 理论、Hermitian 四次和六次理论以及非 Hermitian \( \mathcal{P T} \) 对称三次、四次和五次理论的耦合方程组规模的增加,我们检查了该过程的收敛性和准确性 [7]。截断的 DS 方程为连接的格林函数提供了合理的数值,但这些近似值在高阶检查时不会收敛到准确的结果。



DS 方程直接遵循 \( Z[J] \) [或 \( \log (Z[J])] \) 相对于 \( J \) 的泛函积分,给出 \( \gamma_{n}\left(\right. \) 或 \( \left.G_{n}\right) \)


\[
Z[J]=\int \mathcal{D} \phi \exp \int d x\{-\mathcal{L}[\phi(x)]+J(x) \phi(x)\}
\] 其中 \( \mathcal{L} \) 是拉格朗日量,\( J \) 是 \( c \) 数源,\( Z[0] \) 是欧几里德配分函数 [8,9]。



Hermitian 四次 \( D=0 \) 理论。-泛函积分 \( Z[J] \) 变为普通积分 \( Z[J]= \) \( \int_{-\infty}^{\infty} d \phi e^{-\mathcal{L}(\phi)} \),其中 \( \mathcal{L}(\phi)=\frac{1}{4} \phi^{4}-J \phi \)。精确连通的两点格林函数为




\[
\begin{aligned}
G_{2} & =\int_{-\infty}^{\infty} d \phi \phi^{2} e^{-\phi^{4} / 4} / \int_{-\infty}^{\infty} d \phi e^{-\phi^{4} / 4} \\
& =2 \Gamma\left(\frac{3}{4}\right) / \Gamma\left(\frac{1}{4}\right)=0.675978240 \ldots
\end{aligned}
\]


我们在 \( J=0 \) 时施加奇偶不变性,因此所有奇数格林函数都消失了。第一个非平凡 DS 方程为 \( G_{4}=-3 G_{2}^{2}+1 \)。通过设置 \( G_{4}=0 \) 截断该方程,我们得到近似结果 \( G_{2}=1 / \sqrt{3}=0.577350 \ldots \)。与(1)相比,这个结果低了\( 14.6 \% \),不怎么样。


接下来的三个 DS 方程是




\[
\begin{aligned}
G_{6} & =-12 G_{2} G_{4}-6 G_{2}^{3}, \\
G_{8} & =-18 G_{2} G_{6}-30 G_{4}^{2}-60 G_{2}^{2} G_{4}, \\
G_{10} & =-24 G_{2} G_{8}-168 G_{4} G_{6}-126 G_{2}^{2} G_{6}-420 G_{2} G_{4}^{2} .
\end{aligned}
\]




这个系统是不确定的;未知数的数量总是比方程的数量多一个。为了解决这个系统,我们通过将第一个方程代入第二个方程来消除 \( G_{4} \),我们通过将前两个方程代入第三个方程来消除 \( G_{6} \),依此类推。我们得到 \( G_{2 n} \) 作为 \( n \) 次多项式 \( P_{n}\left(G_{2}\right) \)(除以 \( G_{2} \) 的最高次幂的系数):




\[
\begin{array}{l}
P_{2}(x)=x^{2}-\frac{1}{3}, \quad P_{3}(x)=x^{3}-\frac{2}{5} x \\
P_{4}(x)=x^{4}-\frac{8}{15} x^{2}+\frac{1}{21}, \quad P_{5}(x)=x^{5}-\frac{2}{3} x^{3}+\frac{193}{1890} x .
\end{array}
\]


关闭截断的 DS 方程意味着找到这些多项式的零点。正根绘制在图 1 中。这些根是实根和非退化根,并且向上延伸到 (1) 中的精确 \( G_{2} \)。我们无法先验地知道哪个根最接近 \( G_{2} \),但是根在上端变得更密集,所以我们猜测最大的根给出了最好的近似。



DS 近似值不准确。 -图1中最大根的精度随着截断的顺序缓慢而单调地提高。然而,虽然图 1 中的最大根序列收敛为 \( n \rightarrow \infty \),但极限值为 \( 0.663488 \ldots \),比(1)中的 \( G_{2} \) 的精确值低 \( 1.85 \% \)。出现这种差异是因为截断 DS 方程意味着将 \( G_{2 n} \) 替换为 0 ,但 \( G_{2 n} \) 并不小。 DS 方程是精确的,因此我们可以通过将 (1) 中的 \( G_{2} \) 代入 (3) 来计算 \( G_{2 n} \)。我们发现 Green 的函数随着 \( n \) 快速增长:




\( G_{20}=-4.2788 \times 10^{9}, G_{22}=3.0137 \times 10^{11} \)。 Richardson 外推 [10] 得出 \( G_{2 n} \) 的渐近行为:


\[
G_{2 n} \sim 2 r^{2 n}(-1)^{n+1}(2 n-1) ! \quad(n \rightarrow \infty),
\] 其中 \( r=0.4095057 \ldots \)。


因为 DS 方程在 \( D=0 \) 时是代数的,我们可以解析地推导出这种渐近行为:我们代入 \( G_{2 n}=(-1)^{n+1}(2 n-1) ! g_{2 n} \),将第 \( 2 n \) 个 DS 方程乘以 \( x^{2 n} \),从 \( n=1 \) 到 \( \infty \) 求和,并定义生成函数 \( u(x) \equiv x g_{2}+x^{3} g_{4}+x^{5} g_{6}+\cdots \)。 \( u(x) \)满足的微分方程是非线性的:




\[u^{\prime \prime}(x)=3 u^{\prime}(x) u(x)-u^{3}(x)-x,
\] 其中 \( u(0)=0 \) 和 \( u^{\prime}(0)=G_{2} \)。我们通过代入 \( u(x)=-y^{\prime}(x) / y(x) \) 对 (5) 进行线性化,得到 \( y^{\prime \prime \prime}(x)=x y(x) \),其中 \( y(0)=1, y^{\prime}(0)=0, y^{\prime \prime}(0)=-G_{2} \)。满足这些初始条件的精确解是




\[y(x)=\frac{2 \sqrt{2}}{\Gamma(1 / 4)} \int_{0}^{\infty} d t \cos (x t) e^{-t^{t} / 4} .
\]


如果\( y(x)=0 \),生成函数\( u(x) \)变为无穷大,则\( |x| \)的最小值\( y(x)=0 \)为\( u(x) \)级数的收敛半径。一个简单的图显示 \( y(x) \) 在 \( x_{0}= \pm 2.4419682 \ldots \) [9] 处消失。因此,\( r=1 / x_{0}=0.409506 \ldots \),证实了 (4)。



(4) 中的渐近行为表明 \( G_{2 n} \) 比 \( \gamma_{2 n} \) 的 \( n \rightarrow \infty \) 增长得快得多:




\[
\gamma_{2 n}=\frac{\int_{-\infty}^{\infty} d x x^{2 n} e^{-x^{4} / 4}}{\int_{-\infty}^{\infty} d x e^{-x^{i} / 4}} \sim 2^{n} \frac{\Gamma(n / 2+1 / 4)}{\Gamma(1 / 4)} .
\]


这是惊人的,因为我们通过从 \( \gamma_{2 n} \) 中减去断开连接的部分得到连接的格林函数。
令人惊讶的是,忽略 DS 方程 (3) 左侧的巨大数量 \( G_{2 n} \) 仍然会导致 \( G_{2} \) 的相当准确的结果,如图 1 所示。这是因为虽然左侧的项很大,但右侧的项相对较大 [9]。我们还发现 Padé 近似值或类似平均场的方案不会改善收敛性。但有一种方法可以获得准确的结果:用 (4) 中的渐近公式逼近 DS 方程的左侧,可以得到高精度的 \( G_{2} \)(见图 2)。这种方法适用于 \( D=0 \),但如果 \( D>0 \) 则难以实施,因为 DS 方程是耦合的非线性积分方程而不是代数方程。


非厄米立方 \( D=0 \) 理论。无质量拉格朗日 \( \mathcal{L}=\frac{1}{3} i \phi^{3} \) 定义了一个非厄米 \( \mathcal{P} T \) 对称理论,其单点格林函数为


\[
G_{1}=\int d x x e^{-i x^{3} / 3} / \int d x e^{-i x^{3} / 3},
\],其中积分路径终止于一对 \( \mathcal{P T} \) 对称的 Stokes 扇区 [7],因此 \( G_{1} \) 的精确值为 \( G_{1}=-i 3^{1 / 3} \Gamma\left(\frac{2}{3}\right) / \Gamma\left(\frac{1}{3}\right)=-0.72901113 \ldots i \)。


前四个 DS 方程是




\[
\begin{array}{l}
G_{2}=-G_{1}^{2}, \quad G_{3}=-2 G_{1} G_{2}-i, \\
G_{4}=-2 G_{2}^{2}-2 G_{1} G_{3}, \quad G_{5}=-6 G_{2} G_{3}-2 G_{1} G_{4} .
\end{array}
\]


为了获得 \( G_{1} \) 的主要近似值,我们将第一个方程代入第二个方程并通过设置 \( G_{3}=0 \) 截断。所得方程为 \( G_{1}^{3}=\frac{1}{2} i \),与 \( \mathcal{P T} \) 对称性一致的解为 \( G_{1}=-2^{-1 / 3} i=-0.79370053 \ldots i \)。此结果与 \( G_{1} \) 的确切值相差 \( 8.9 \% \)。

在高阶,我们再次截断系统并在 \( G_{1} \) 中找到相关多项式的根。首先,与通过此过程获得的 \( \mathcal{P T} \) 对称性一致的根接近精确的 \( G_{1} \),但与 Hermitian 四次理论的根不同,后者的方法是单调的(图 1),该方法是振荡的:对于 \( n=4,5,6,7 \) 截断最接近的根确切的 \( G_{1} \) 是 \( -0.693361 \ldots i \)、\( -0.746900 \ldots i, \quad-0.712564 \ldots i, \quad \) 和 \( -0.739871 \ldots i \)。然而,对于 \( n=8 \),这种模式被打破了;最近的根是 \( -0.712368 \ldots i \),它比 \( n=6 \) 根更差。

这种偏离振荡收敛是近似值发生质变的第一个迹象。对于 \( n=10 \),最靠近 \( G_{1} \) 的根是位于 \( -0.717367 \ldots i \pm 0.016050 \ldots \) 负虚轴两侧的一对。我们求解 DS 方程直到第 150 次截断,并在图 3 中绘制从 \( n=2 \) 到 150 的所有根作为复平面中的点。这些根在三叶螺旋桨形状上变得密集,每个叶片的尖端都有一个小环。插图显示环上的点环绕但不接近确切的 \( G_{1} \)。

图 3 中的根具有三重对称性,因为截断的 DS 方程给出的多项式仅具有 \( x^{3} \) 的幂(除了 0 处的根)。 DS 方程仅局部依赖于泛函积分的被积函数;它们对函数积分的边界条件完全不敏感。存在三对开角为 \( 60^{\circ} \) 的 Stokes 扇区,式 (7) 中的积分路径可以在其中终止。这些扇区以 \( \theta_{1}=(\pi / 2), \theta_{2}=-(\pi / 6) \) 或 \( \theta_{3}=-(5 \pi / 6) \) 为中心。如果积分路径终止于 \( \mathcal{P} T \) 对称 \( (2,3) \) 扇区,则 \( G_{1} \) 为负虚数,但如果它终止于 \( (1,2) \) 或 \( (1,3) \) 扇区,则 \( G_{1} \) 为复数。


\( G_{n} \) 对于大 n.-Richardson 外推的渐近行为给出了三次理论 \( \left(G_{14}=42692.806116\right. \), \( G_{15}=-255589.034701 i \) 的精确格林函数的大 \( n \) 行为:


\[
G_{n} \sim-(n-1) ! r^{n}(-i)^{n} \quad(n \rightarrow \infty),
\] 其中 \( r=0.427696347707 \ldots \)。等式 (9) 类似于 Hermitian 四次理论的 (4),并且可以通过分析得到证实 [9]。



为了分析计算 \( r \),我们遵循上面用于 Hermitian 四次理论的程序。定义 \( g_{p} \equiv-i^{n} G_{p} /(p-1) \) !并将格林函数 \( G_{n} \) 的 DS 方程重写为一个紧凑方程:




\[g_{p}=\frac{1}{p-1} \sum_{k=1}^{p-1} g_{k} g_{p-k}+\frac{1}{2} \delta_{p, 3} \quad(p \geq 2) .
\]


接下来,乘以 \( (p-1) x^{p} \),将 \( p=2 \) 与 \( \infty \) 相加,定义生成函数 \( f(x) \equiv \sum_{p=1}^{\infty} x^{p} g_{p} \),它服从 Riccati 方程 \( x f^{\prime}(x)-f(x)=f^{2}(x)+x^{3} \)。


代入 \( f(x)=-x u^{\prime}(x) / u(x) \) 可线性化此等式:\( u^{\prime \prime}(x)=-x u(x) \)。这是一个 Airy 方程,其通解为 \( u(x)=a \mathrm{Ai}(-x)+b \mathrm{Bi}(-x) \)。从


\( f^{\prime}(0)=g_{1}=-3^{1 / 3} \Gamma\left(\frac{2}{3}\right) / \Gamma\left(\frac{1}{3}\right) \)我们发现\( a \)是任意的,\( b=0 \),所以\( f(x)=x \mathrm{Ai}^{\prime}(-x) / \operatorname{Ai}(-x) \)。

当分母消失时,生成函数 \( f(x) \) 的幂级数在 \( x=2.338107410460 \ldots \) 时爆炸。这是级数的收敛半径,它的倒数是 (9) 中 \( r \) 的值。

(9) 中 \( G_{n} \) 的快速增长解释了通过截断 DS 方程获得的缓慢收敛和不准确的数值结果(图 3)。再一次,使用这个渐近近似而不是设置 \( G_{n}=0 \) 给出了对 \( G_{1}[9] \) 的极其准确和快速收敛的近似。



非厄米四次 \( D=0 \) 理论。拉格朗日 \( \mathcal{L}=-\frac{1}{4} \phi^{4} \) 定义了非厄米 \( \mathcal{P} T \) 对称理论,其中


\[
G_{1}=\frac{\int d x x \exp \left(x^{4} / 4\right)}{\int d x \exp \left(x^{4} / 4\right)}=-\frac{2 i \sqrt{\pi}}{\Gamma(1 / 4)}=-0.977741 \ldots i,
\] 并且集成路径位于下半复合体 \( x \) 平面中的 \( \mathcal{P} T \) 对称斯托克斯扇区对内。


前三个 DS 方程是




\[
\begin{aligned}
G_{3} & =-G_{1}^{3}-3 G_{1} G_{2} \\
G_{4} & =-3 G_{1} G_{3}-3 G_{2}^{2}-3 G_{1}^{2} G_{2}-1, \\
G_{5} & =-3 G_{1} G_{4}-9 G_{2} G_{3}-3 G_{1}^{2} G_{3}-6 G_{1} G_{2}^{2}
\end{aligned}
\]


求解这些方程比厄米四次或非厄米三次理论更难,因为必须将两个格林函数设置为零才能关闭系统,并且必须同时求解两个耦合多项式方程。前导顺序截断导致 \( G_{1}=-i(3 / 2)^{1 / 4}=-1.106682 \ldots i \),它与上面的 \( 13.2 \% \) 完全不同。

对于较大的 \( n \),将继续此过程。根的数量迅速增加,并且根在复平面上具有四重对称性。直到 \( n=33 \) 的所有根如图 4 所示。轴上有四个根集中,但 \( \mathcal{P T} \) 对称性要求 \( G_{1} \) 为负虚数。与图 3 不同的是,点散布在复平面上,因为截断 DS 方程会给出两个耦合多项式方程。

我们可以从 (10) 中的 DS 方程确定 \( G_{n} \) 对于大 \( n \) 的渐近行为。我们找到 \( G_{n} \sim-i(n-1) !(-i)^{n} r^{n} \),其中 \( r=0.34640 \ldots \) 此结果类似于 (9) 中的渐近行为。


五次和六次 \( D=0 \) 理论。-\( \mathcal{P} T \) 对称 \( D=0 \) 拉格朗日 \( -\frac{1}{5} i \phi^{5} \) 的 DS 方程需要将三个更高的格林函数设置为 0 以关闭截断系统,从而导致 \( G_{1}, G_{2} \) 和 \( G_{3} \) 的三个耦合多项式方程。转到 \( n=11 \) 截断,我们在图 5 中看到十个根浓度。(DS 方程对函数积分的 Stokes 扇区的选择不敏感。)有两对


\( \mathcal{P} T \) 对称边界条件,产生两个虚值 \( G_{1}=0.412009 \ldots i \) 和 \( G_{1}= \) \( -1.078653 \ldots i \) [11],在图 5 中显示为重点。

对于六列情况 \( \mathcal{L}=\frac{1}{6} \phi^{6} \),我们截断 DS 方程并将四个最高的格林函数设置为 0 。我们必须求解四个耦合多项式方程。为了减少解决方案的数量,我们施加了奇偶校验对称性,所以 \( G_{1}=G_{3}=0 \)。这消除了除三对 Stokes 扇区之外的所有扇区。图 6 显示了 \( G_{2} \) 到 \( n=32 \) 截短的三种根浓度。 \( G_{2} \)(正方形)的确切值为\( 6^{1 / 3} \sqrt{\pi} / \Gamma(1 / 6)=0.578617 \ldots \)和\( -0.289302 \ldots \pm 0.501097 i \);误差是百分之几。
总结.-对于五个 \( D=0 \) 场论,我们已经证明截断的 DS 方程产生欠定多项式系统。没有解决此类系统的有效策略:通过将更高的格林函数设置为零来关闭系统会给出收敛到不正确极限值的近似值序列。用类似平均场的近似值替换更高的格林函数也会给出不正确的极限值,并且这种方法的缺点是如果 \( D>0 \),则需要重新归一化。一种数值上准确的方法是用较大的 \( n \) 渐近行为替换较高的 \( G_{n} \) 。这在 \( D>0 \) 时是困难的,但我们相信计算是可能的,它为进一步研究提供了一条有趣的途径。

这封信强调 DS 方程是局部的。推导 DS 方程仅假设函数积分存在; DS 方程对使用函数空间中的哪些 Stokes 扇区不敏感。结果,近似值尝试(但未能)接近许多不同的限制,其中大部分都很复杂 [12]。

当相互作用项具有更高的场次幂时,DS 截断的准确性会变差,因为系统的不确定性会增加。更多格林函数必须设置为0才能关闭截断系统。


对于具有弱耦合常数 \( g \) 的拉格朗日量,我们可以将 DS 方程中的所有 \( G_{n} \) 展开为以下的幂级数


\( g \)。这消除了此处讨论的所有歧义,并为每个 \( G_{n} \) 提供了独特的弱耦合扩展。然而,这仅仅复制了格林函数的费曼图计算,完全忽略了理论的非微扰内容。

C. M. B. 感谢亚历山大·冯·洪堡和西蒙斯基金会,以及英国工程和物理科学研究委员会提供的财政支持。



\end{document}

